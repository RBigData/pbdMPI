
{\color{red} \bf Warning:}
The findings and conclusions in this article have not been
formally disseminated by the U.S. Department of Energy
and should not be construed to represent any determination or
policy of University, Agency, and National Laboratory.

This document is written to explain the main
functions of \pkg{pbdMPI}~\citep{Chen2012}, version 0.1-1.
Every effort will be made to ensure future versions are consistent with
these instructions, but features in later versions may not be explained
in this document.

Information about the functionality of this package,
and any changes in future versions can be found on website:
``Programming with Big Data in R'' at
\url{http://r-pbd.org/}.
%More discussion about parallel computing in \proglang{R}~\citep{Rcore}
%can be found in~\citet{Schmidberger2009}.




\section[Introduction]{Introduction}
\label{sec:introduction}
\addcontentsline{toc}{section}{\thesection. Introduction}

Our intent is to bring the most common parallel programming model from
supercomputing, Single Program Multiple Data (SPMD), to \proglang{R}
and enable distributed handling of truly large data. Consequently,
\pkg{pbdMPI} is intended for batch mode programming with big data
(pbd). Unlike \pkg{Rmpi}~\citep{Yu2010},
\pkg{snow}~\citep{Tierney2012}, or \pkg{parallel}~\citep{Rcore},
interactive mode is not supported.  We think that interaction with a
large distributed parallel computing platform is better handled with a
client/server relationship, and we are developing other packages in
this direction.  \pkg{pbdMPI} simplifies MPI interaction, but leaves
low and mid level functions available for advanced programmers. For
example, it is easy to hand communicators to \pkg{pbdMPI} from other
applications through MPI array pointers. This is intended for
integration with other, possibly non-R, parallel software.

% Frankly, the interactive mode is proper for small data or short loops, but
% could take long time for waiting next responding,
% especially tedious tasks or a large number of tasks.
% One never know when to input or type next commands.
% Once one needs parallel computing, which means the computing time is
% sufficiently longer than what one usually expect for interaction with
% machines. On the other hand,
% for well defined scripts, it is always efficient to run
% in batch instead of interactive. Therefore, the SPMD programming model
% is considerable easier than manager/worker programming model.
% \pkg{pbdMPI} provides a better way for communications
% between processors/machines.

Under the SPMD parallel programming model, the identical program runs
on every processor but typically works on different parts of a large
data set, while communicating with other copies of itself as
needed. Differences in execution stem from \code{comm.rank}, which is
typically different on every processor. While on the surface this
sounds complicated, after some experience and a new mindset,
programming is surprisingly simple. There is no master. There is only
cooperation among the workers. Although we target very large
distributed computing platforms, SPMD works well even on small
multicore platforms.

In the following, we list the main features of \pkg{pbdMPI}.
\begin{enumerate}
\item Under the SPMD batch programming model, a single program is
  written, which is spawned by \code{mpirun}. No spawning
  and broadcasting from within \proglang{R} are required.
% \item Preserve original \proglang{R} style,
%       i.e. Easy to modify/parallelize serial \proglang{R} codes.
\item S4 methods are used for most collective functions so it is easy
  to extend them for general \proglang{R} objects.
\item Default methods (like \code{Robj} functions in \pkg{Rmpi}) have
  homogeneous checking for data type so they are safe for general
  users.
\item The API in all functions is simplified, with all default arguments
  in control objects.
\item Methods for array or matrix types are implemented without
  serialization and un-serialization, resulting in faster
  communication than \pkg{Rmpi}.
\item Basic data types of integer, double and raw in \pkg{pbdMPI} are
  communicated without further checking. This is risky but fast for
  advanced programmers.
\item Character data type is serialized and communicated by raw type.
\end{enumerate}

System requirements and installation of \pkg{pbdMPI} are described
next.  Section~\ref{sec:performance} gives a short example for
comparing performance of \pkg{pbdMPI} and \pkg{Rmpi}~\citep{Yu2010}.
In Section~\ref{sec:faqs}, a few quick answers for questions are
given.  Section~\ref{sec:windows_systems} provides settings for
Windows environments.  Finally, in Section~\ref{sec:analog_examples},
two examples from \pkg{parallel} are shown as SPMD \pkg{pbdMPI}
programs.

\subsection[System Requirements]{System Requirements}
\label{sec:system_requirements}
\addcontentsline{toc}{subsection}{\thesubsection. System Requirements}

\pkg{pbdMPI} requires MPI
(\url{http://en.wikipedia.org/wiki/Message_Passing_Interface}). The
package is mainly developed and tested under {\color{blue} OpenMPI}
(\url{http://www.open-mpi.org/}) in xubuntu 11.04
(\url{http://xubuntu.org/}) and should also work with LAM/MPI
(\url{http://www.lam-mpi.org/}) and MPICH2
(\url{http://www.mcs.anl.gov/research/projects/mpich2/}).  In addition
to unix, \pkg{pbdMPI} should also run under other operating systems
such as Mac OS X with OpenMPI or Windows 7 with MPICH2 if MPI is
installed and launched properly, although we have not tested on
multiple machines yet.  Please let us know about your experience.

For normal installation, see Sec.~\ref{sec:installation}. To build as
a static library, which may be required on some large systems, use
\begin{Command}
./configure --enable-static --prefix=${MPI_ROOT}
make
make install
\end{Command}
where {\color{red} \code{-}\code{--enable-static}} can build a static
library (optional), and \code{$\{MPI_ROOT\}} is the path to MPI root.
Note that the static library is not necessary for \pkg{pbdMPI} but may
avoid dynamic loading problems.

To make sure your MPI system is working, test with
\begin{Command}
mpiexec -np 2 hostname
\end{Command}
This should list two host names where MPI jobs are running.  Note to
use \code{hostname.exe} with the extension on a Windows system.


\subsection[Installation and Quick Start]{Installation and Quick Start}
\label{sec:installation}
\addcontentsline{toc}{subsection}{\thesubsection. Installation and Quick Start}

One can download \pkg{pbdMPI} from CRAN at
\url{http://cran.r-project.org}, and
the intallation can be done with the following commands
(using OpenMPI library)
\begin{Command}
tar zxvf pbdMPI_0.1-0.tar.gz
R CMD INSTALL pbdMPI
\end{Command}
Further configure arguments include
\\
\begin{center}
\vspace{0.2cm}
\begin{tabular}{ll} \hline\hline
Argument                   & Default \\ \hline
\code{-}\code{--with-Rmpi-type}    & \code{OPENMPI} \\
\code{-}\code{--with-Rmpi-include} & \code{$\{MPI_ROOT\}/include} \\
\code{-}\code{--with-Rmpi-libpath} & \code{$\{MPI_ROOT\}/lib} \\
\code{-}\code{--with-mpi}          & \code{$\{MPI_ROOT\}} \\
\hline\hline
\end{tabular}
\vspace{0.2cm}
\end{center}
where \code{$\{MPI_ROOT\}} is the path to the MPI root.  For
non-default and unusual installations of MPI systems, the commands may
be
\begin{Command}
### Under command mode
R CMD INSTALL pbdMPI \
  --configure-args="--with-Rmpi-type=OPENMPI \
                    --with-mpi=/usr/local"
R CMD INSTALL pbdMPI \
  --configure-args="--with-Rmpi-type=OPENMPI \
                    --with-Rmpi-include=/usr/local/ompi/include \
                    --with-Rmpi-libpath=/usr/local/ompi/lib"
\end{Command}
See the package source file \code{pbdMPI/configure} for details.

One can get started quickly with \pkg{pbdMPI} by learning from the
following six examples.
\begin{Command}
### At the shell prompt, run the demo with 2 processors by
### (Use Rscript.exe for windows system)
mpiexec -np 2 Rscript -e "demo(allgather,'pbdMPI',ask=F,echo=F)"
mpiexec -np 2 Rscript -e "demo(allreduce,'pbdMPI',ask=F,echo=F)"
mpiexec -np 2 Rscript -e "demo(bcast,'pbdMPI',ask=F,echo=F)"
mpiexec -np 2 Rscript -e "demo(gather,'pbdMPI',ask=F,echo=F)"
mpiexec -np 2 Rscript -e "demo(reduce,'pbdMPI',ask = F,echo=F)"
mpiexec -np 2 Rscript -e "demo(scatter,'pbdMPI',ask=F,echo=F)"
\end{Command}


\subsection[Basic Steps]{Basic Steps}
\label{sec:basic_steps}
\addcontentsline{toc}{subsection}{\thesubsection. Basic Steps}

In the SPMD world, every processor is a worker, every worker knows
about all the others, and each worker does its own job, possibly
communicating with the others. Unlike the manager/workers style, SPMD
is more likely to fully use the computer resources.  The following
shows typical basic steps of using \pkg{pbdMPI}.
\begin{enumerate}
\item Initialize. (\code{init})
\item Read your portion of the data.
\item Compute. (\code{send}, \code{recv}, \code{barrier}, ...)
\item Communicate results among workers.
(\code{gather}, \code{allgather}, \code{reduce}, \code{allreduce}, ...)
\item Finalize. (\code{finalize})
\end{enumerate}
In a given application, the Compute and Communicate steps may be
repeated several times for intermediate results. The Compute and
Communicate steps are more general than the ``map'' and ``reduce''
steps of the map-reduce paradigm but similar in spirit.  One big
difference is that the Communicate step may place the ``reductions''
on all processors rather than just one (the manager for map-reduce)
for roughly the same time cost.  With some experience, one can easily
convert existing \proglang{R} scripts, and quickly parallelize serial
code. \pkg{pbdMPI} tends to reduce programming effort, avoid
complicated MPI techniques, and gain computing performance.

The major communication functions of \pkg{pbdMPI} and corresponding
similar functions of \pkg{Rmpi} are listed in the following.
\\
\begin{center}
\vspace{0.2cm}
\begin{tabular}{cl} \hline\hline
\pkg{pbdMPI} (S4) & \pkg{Rmpi}                \\ \hline
\code{allgather}    & \code{mpi.allgather},
                      \code{mpi.allgatherv},
                      \code{mpi.allgather.Robj} \\
\code{allreduce}    & \code{mpi.allreduce}      \\
\code{bcast}        & \code{mpi.bcast},
                      \code{mpi.bcast.Robj}     \\
\code{gather}       & \code{mpi.gather},
                      \code{mpi.gatherv},
                      \code{mpi.gather.Robj}    \\
\code{recv}         & \code{mpi.recv},
                      \code{mpi.recv.Robj}      \\
\code{reduce}       & \code{mpi.reduce}         \\
\code{scatter}      & \code{mpi.scatter},
                      \code{mpi.scatterv},
                      \code{mpi.scatter.Robj}   \\
\code{send}         & \code{mpi.send},
                      \code{mpi.send.Robj}      \\ \hline \hline
\end{tabular}
\end{center}


\subsection[More Examples]{More Examples}
\label{sec:more_examples}
\addcontentsline{toc}{subsection}{\thesubsection. More Examples}

The package source files provide several examples based on \pkg{pbdMPI},
such as \\
\begin{center}
\vspace{0.2cm}
\begin{tabular}{ll} \hline\hline
Directory & Examples \\ \hline
\code{pbdMPI/inst/examples/test_spmd/}         & main SPMD functions \\
\code{pbdMPI/inst/examples/test_rmpi/}         & comparison to \pkg{Rmpi} \\
\code{pbdMPI/inst/examples/test_parallel/}     & comparison to \pkg{parallel} \\
\code{pbdMPI/inst/examples/test_performance/}  & performance testing \\
\code{pbdMPI/inst/examples/test_s4/}           & S4 extension \\
\hline\hline
\end{tabular}
\end{center}


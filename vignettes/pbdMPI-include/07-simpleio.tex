
\section[Simple Input and Output]{Simple Input and Output}
\label{sec:simple_io}
\addcontentsline{toc}{section}{\thesection. Simple Input and Output}

We add new supports simple data input and output for basic CSV and text
files to \pkg{pbdMPI} since version 0.2-2.

Two quick demos can simply explain how a dataset can be input and output
via \pkg{pbdMPI} functions \code{comm.write.table()} and
\code{comm.read.table()}. The first is
\begin{Command}
### Run the demo with 4 processors by
mpiexec -np 4 Rscript -e "demo(simple_io,'pbdMPI',ask=F,echo=F)"
\end{Command}
The demo utilizing \code{iris} data~\citep{Fisher1936} to show simple input
and output functions of \pkg{pbdMPI} and is summarized as in next.
\begin{itemize}
\item 150 rows of \code{iris} are divided in 4 processors, and processors
      own 37, 37, 38, and 38 rows of \code{iris} as a gbd row-block format.
      i.e. Rank 0 owns row 1 to 37, rank 1 owns row 38 to 74, and so on.
\item A text file ``iris.txt'' is dumped via \code{comm.write.table()} which
      sequentially append processor owned row blocks.
\item \code{comm.read.table()} then reads the text file back in memory, and
      again in a gbd row-block format.
\end{itemize}

Note that \code{comm.read.table()} may read a first few lines to predetermine
how many lines of the file to read in. This is an approximation and results
in unbalance data across processors. In particular, either the highest order
rank may own the largest portion of whole dataset, or several higher order
ranks may own zero row. So, a call
\code{comm.load.balance()} within \code{comm.read.table()}
is to move rows across processors if necessary.
Basically, the reading steps are described as in the next.
\begin{enumerate}
\item If file size were less than 5MB, then rank 0 would read in the whole file
      and scatter rows to other ranks.
\item If file size were large than 5MB, then rank 0 would read in the first
      500 lines and estimate total number of records in the file. All ranks
      sequentially read in the designated records.    
\item Call \code{comm.load.balance()} to balance the data.
\end{enumerate}
The file size limit is controlled by \code{.pbd_env$SPMD.IO$max.file.size},
and the first few line limit is controlled by
\code{.pbd_env$SPMD.IO$max.test.lines}.
Further, users can specify options \code{nrows} and \code{skip} to
\code{comm.read.*()} to manually read the file and call
\code{comm.load.balance()} later if needed.

There are several way to distributed or balance data among processors.
Currently \pkg{pbdMPI} supports 3 formats: \code{block}, \code{block0}, and
\code{block.cyclic}. In the above demo, the 150 rows are mainly distributed in
$(37, 37, 38, 38)$ which is a \code{block} format. The second demo shows
how to load balance between different formats next.
\begin{Command}
### Run the demo with 4 processors by
mpiexec -np 4 Rscript -e "demo(simple_balance,'pbdMPI',ask=F,echo=F)"
\end{Command}
In the \code{block0}, the \code{iris} is distributed as $(38, 38, 37, 37)$
row-bock of each processor.
In the \code{block.cyclic}, the \code{iris} is distributed as
$(38, 38, 38, 36)$ row-bock of each processor. i.e. Each cycle has 38
rows and one cycle per processor.

See \pkg{pbdDEMO} vignettes~\citep{Schmidt2013pbdDEMOvignette}
for more details about ``block-cyclic'' and ``gbd''.

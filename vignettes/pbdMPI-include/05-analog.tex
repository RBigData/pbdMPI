
\section[SPMD in Examples from package \pkg{parallel}]{SPMD in Examples from package \pkg{parallel}}
\label{sec:analog_examples}
\addcontentsline{toc}{section}{\thesection. SPMD in Examples from package \pkg{parallel}}

We demonstrate how a simple script from \pkg{parallel} can be
written in batch by using \pkg{pbdMPI}.  Each time, we first give the
version using \pkg{parallel} followed by the
version using \pkg{pbdMPI}. All codes are available in
\code{pbdMPI/inst/examples/test_parallel/}. \\

{\bf Example 1:}
(\code{mclapply()} originates in \pkg{multicore}~\citep{Urbanek2011}) \\
Save the following script in a file and run with
\begin{Command}
Rscript 01_mclapply_par.r
\end{Command}
to see the computing time on your platform.
\begin{Code}[title=\pkg{multicore} R Script]
### File Name: 01_mclapply_par.r
library(parallel)

system.time(
  unlist(mclapply(1:32, function(x) sum(rnorm(1e7))))
)
\end{Code}
Now save this script in a file and run with
\begin{Command}
mpirun -np 2 Rscript 01_mclapply_spmd.r
\end{Command}
to see the computing time on your platform.
\begin{Code}[title=SPMD R Script]
### File Name: 01_mclapply_spmd.r
library(pbdMPI, quiet = TRUE)
init()

time.proc <- system.time({
  id <- get.jid(32)
  ret <- unlist(lapply(id, function(i) sum(rnorm(1e7))))
  ret <- allgather(ret, unlist = TRUE)
})
comm.print(time.proc)

finalize()
\end{Code}

The following shows the computing time of the above codes
on a single local machine with two cores
Intel(R) Core(TM) i5-2410M CPU @ 2.30 GHz, xubuntu 11.04 system,
and OpenMPI 1.6. There is not much communication latency in this example
since all computings are on one ``node'' which is also
a limitation of \pkg{parallel}.
\begin{CodeOutput}
>> Test ./01_mclapply_par.r
   user  system elapsed
 16.800   0.570  17.419

>> Test ./01_mclapply_spmd.r
COMM.RANK = 0
   user  system elapsed
 17.130   0.460  17.583
\end{CodeOutput}


{\bf Example 2:} (\code{parMM()} originates in \pkg{snow}~\citep{Tierney2012}) \\
Save the following code in a file and run with two processors
\begin{Command}
Rscript 02_parMM_par.r
\end{Command}
to see the computing time on your platform.
\begin{Code}[title=\pkg{snow} R Script]
### File Name: 02_parMM_par.r
library(parallel)
cl <- makeCluster(2)

splitRows <- function (x, ncl){
  lapply(splitIndices(nrow(x), ncl), function(i) x[i, , drop = FALSE])
}
parMM <- function (cl, A, B){
  do.call(rbind, clusterApply(cl, splitRows(A, length(cl)), get("%*%"), B))
}

set.seed(123)
A <- matrix(rnorm(1000000), 1000)
system.time(replicate(10, A %*% A))
system.time(replicate(10, parMM(cl, A, A)))

stopCluster(cl)
\end{Code}
Now save this script in a file and run with
\begin{Command}
mpirun -np 2 Rscript 02_parMM_spmd.r
\end{Command}
to see the computing time on your platform.
\begin{Code}[title=SPMD R Script]
### File Name: 02_parMM_spmd.r
library(pbdMPI, quiet = TRUE)
init()

set.seed(123)
x <- matrix(rnorm(1000000), 1000)

parMM.spmd <- function(x, y){
  id <- get.jid(nrow(x))
  do.call(rbind, allgather(x[id,] %*% y))
}
time.proc <- system.time(replicate(10, parMM.spmd(x, x)))
comm.print(time.proc)

finalize()
\end{Code}

The following shows the computing time of the above code
on a single machine with two processors
Intel(R) Core(TM) i5-2410M CPU @ 2.30 GHz, xubuntu 11.04 system,
and OpenMPI 1.6. \pkg{pbdMPI} performs better than \pkg{snow} in this
example even without communication over network.
\begin{CodeOutput}
>> Test ./02_parMM_par.r
   user  system elapsed 
 12.460   0.170  12.625 
   user  system elapsed 
  1.780   0.820  10.095 

>> Test ./02_parMM_spmd.r
COMM.RANK = 0
   user  system elapsed 
   8.84    0.42    9.26 
\end{CodeOutput}



\section[Windows Systems (MS-MPI)]{Windows Systems (MS-MPI)}
\label{sec:windows_systems_msmpi}
\addcontentsline{toc}{section}{\thesection. Windows Systems (MS-MPI)}

Originally, \pkg{pbdMPI} (later than version 0.2-3 but only up to version 0.3-1)
supports Windows with Microsoft MPI or MS-MPI
(\url{http://msdn.microsoft.com/en-us/library/bb524831(v=vs.85).aspx}).
\pkg{pbdMPI} was built with
'HPC Pack 2012 R2 MS-MPI Redistributable Package' which is available
at \url{http://http://www.microsoft.com/en-us/download/}.
The installation (MSMPISetup.exe) is easily done with a few clicks
provided some service packs and Visual C++ runtime are installed correctly.
The default environment and path are recommended for installation.

Currently, \pkg{pbdMPI} (later than version 0.3-2) supports Windows with
Microsoft MPI or MS-MPI version 7.1
(\url{https://www.microsoft.com/en-us/download/details.aspx?id=52981}).
Note that this is only a SDK development library which does not contain
any MPI executable file such as \code{mpiexec.exe}. This is only for
compiling and linking the \pkg{pbdMPI} with MPI library.
However, you will still need 'MS-MPI Redistributable Package' to have a
\code{mpiexec.exe} to run MPI programs or \pkg{pbdMPI} scripts.

The difference of default installation between the SDK library and
'Redistributable Package' are
\begin{itemize}
\item
the location of MPI header file, and
\item
the location of the default installation.
\end{itemize}
The include path is changed from (Redistributable)
\begin{Command}
MPI_INCLUDE = ${MPI_ROOT}Inc/
\end{Command}
to (SDK)
\begin{Command}
MPI_INCLUDE = ${MPI_ROOT}Include/
\end{Command}
These are used by \code{pbdMPI/src/Makevars.win}.
The default installation is changed from (Redistributable)
\begin{Command}
SET MPI_HOME=C:\Program Files\Microsoft MPI\
\end{Command}
to (SDK)
\begin{Command}
SET MPI_HOME=C:\Program Files (x86)\Microsoft SDKS\MPI\
\end{Command}
These are supposed to be set in a batch file.


For running MPI and \proglang{R}, users need to set \code{PATH} to the
\code{mpiexec.exe} and \code{Rscript.exe}. By default,
\begin{Command}
### Under command mode, or save in a batch file.
SET R_HOME=C:\Program Files\R\R-3.0.1\
SET MPI_HOME=C:\Program Files\Microsoft MPI\
SET PATH=%MPI_HOmE%bin\;%R_HOME%bin\;%PATH%
\end{Command}

Note that the installation (MSMPISetup.exe) may set
several environmental variables including
\begin{itemize}
\item \code{MSMPI_BIN} for \code{mpiexec.exe} and other executable files,
\item \code{MSMPI_INC} for header files such as \code{mpi.h},
\item \code{MSMPI_LIB32} for 32 bits static libraries such as \code{msmpi.lib},
      and
\item \code{MSMPI_LIB64} for 64 bits static libraries.
\end{itemize}
These should be useful to verify via \proglang{R} command \code{Sys.getenv()}.


\subsection[Install from Binary]{Install from Binary}
\label{sec:install_from_binary}
\addcontentsline{toc}{subsection}{\thesubsection. Install from Binary}

The binary packages of \pkg{pbdMPI} are available on the website:
``Programming with Big Data in R'' at
\url{http://r-pbd.org/} or
``CRAN'' at
\url{http://cran.r-project.org/package=pbdMPI}.
Note that different MPI systems require different binaries.
The binary can be installed by
\begin{Command}
R CMD INSTALL pbdMPI_0.2-3.zip
\end{Command}

As on Unix systems,
one can start quickly with \pkg{pbdMPI} by learning from the
following demos. There are six basic examples.
\begin{Command}
### Run the demo with 2 processors by
mpiexec -np 2 Rscript.exe -e "demo(allgather,'pbdMPI',ask=F,echo=F)"
mpiexec -np 2 Rscript.exe -e "demo(allreduce,'pbdMPI',ask=F,echo=F)"
mpiexec -np 2 Rscript.exe -e "demo(bcast,'pbdMPI',ask=F,echo=F)"
mpiexec -np 2 Rscript.exe -e "demo(gather,'pbdMPI',ask=F,echo=F)"
mpiexec -np 2 Rscript.exe -e "demo(reduce,'pbdMPI',ask=F,echo=F)"
mpiexec -np 2 Rscript.exe -e "demo(scatter,'pbdMPI',ask=F,echo=F)"
\end{Command}
{\color{red}Warning:}
Note that spacing inside \code{demo} is not working for Windows systems
and \code{Rscript.exe} should be evoked rather than \code{Rscript}.


\subsection[Build from Source]{Build from Source}
\label{sec:building_from_source}
\addcontentsline{toc}{subsection}{\thesubsection. Build from Source}

{\color{red} \bf Warning:} This section is only for building binary in
32- and 64-bit Windows system. A more general way can be found in the file
\code{pbdMPI/INSTALL}.

Make sure that \proglang{R}, \pkg{Rtools}, and \pkg{MINGW} are in
the \code{PATH}.
See details on the website "Building R for Windows" at
\url{http://cran.r-project.org/bin/windows/Rtools/}.
The environment variable \code{MPI_HOME}
needs to be set for building binaries.

For example, the minimum requirement (for Rtools32 or earlier) may be
\begin{Command}
### Under command mode, or save in a batch file.
SET R_HOME=C:\Program Files\R\R-3.0.1\
SET RTOOLS=C:\Rtools\bin\
SET MINGW=C:\Rtools\gcc-4.6.3\bin\
SET MPI_HOME=C:\Program Files\Miscrosoft MPI\
SET PATH=%MPI_HOME%bin;%R_HOME%;%R_HOME%bin;%RTOOLS%;%MINGW%;%PATH%
\end{Command}

For example, the minimum requirement (for Rtools33 or later) may be
\begin{Command}
### Under command mode, or save in a batch file.
SET R_HOME=C:\Program Files\R\R-3.4.0\
SET RTOOLS=C:\Rtools\bin\
SET MPI_HOME=C:\Program Files\Miscrosoft MPI\
SET PATH=%MPI_HOME%bin;%R_HOME%;%R_HOME%bin;%RTOOLS%;%PATH%
\end{Command}
Note that \code{gcc} and others within \pkg{Rtools} will be detected by
windows \proglang{R}, so the installation path of \pkg{Rtools} should be
exactly the same as \code{C:/Rtools}.

With a correct \code{PATH}, one can use the \proglang{R} commands
to install/build the \pkg{pbdMPI}:
\begin{Command}
### Under command mode, build and install the binary.
tar zxvf pbdMPI_0.2-3.tar.gz
R CMD INSTALL --build pbdMPI
R CMD INSTALL pbdMPI_0.2-3.zip
\end{Command}

{\color{red} \bf Warning:} For other \proglang{pbdR} packages, it is possible
to compile without further changes of configurations. However, only
\pkg{pbdMPI} is tested regularly before any release.


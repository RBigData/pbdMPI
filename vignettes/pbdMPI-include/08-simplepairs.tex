
\section[Simple Pairwise Evaluation]{Simple Pairwise Evaluation}
\label{sec:simple_pairs}
\addcontentsline{toc}{section}{\thesection. Simple Pairwise Evaluation}

We build some utilities for pairwise evaluation to
\pkg{pbdMPI} since version 0.2-3.

Evaluating a function on any two data points is a common problems, such as
distance, pairwise comparison, and multiple testing problems.
Useful functions to solve those problems are
\begin{itemize}
\item
\code{comm.as.gbd()}:
This function is to turn a common matrix (in all ranks)
to a gbd matrix in row major blocks. For example, one may read in data
from one rank, then utilizes this function to redistribute data with load
balance of all ranks. This is an alternative way to
Section~\ref{sec:simple_io}, but more efficient for small size of data.

\item
\code{comm.allpairs()}:
This function is mainly to provide indices for all pairs of $N$ data points.
It returns a two columns (i, j)  gbd matrix in row major blocks.
For example, one may want to evaluate all $N^2$ pairs of the $N$ data points.
However, in distance context, it provides only indices as in lower-triangular
matrix (ordered by row major).

\item
\code{comm.dist()}:
This function is to compute distance (lower-triangular only) of $N$ data points
as usual \code{dist()} function, but evaluated on a gbd matrix in row major
blocks. The returning can be a common distance matrix (only good for small
dataset), or a 3 columns gbd matrix in row major blocks. The columns are
i, j, and the value of pair (i, j).

\item
\code{comm.pairwise()}:
This functions is a general extension composed of three functions above
that allows
users to provide a function \code{FUN} to evaluate on pairs of data.
For example, a distance between two data points \code{x} and \code{y}
can be computed via original \code{dist()} function. So, it can be wrapped as
\begin{Code}
dist.pair <- function(x, y, ...){
  as.vector(dist(rbind(x, y), ...))
}
\end{Code}
for the \code{FUN} option of \code{comm.pairwise()}.

This function is also useful for cases that measure of pair (i, j)
differs to that of pair (j, i), i.e. non-symmetric measure.
If order is matter, then the \code{FUN}
can be evaluated via the options either \code{pairid.gbd}
which can be user defined or simply \code{symmetric = FALSE}.
\end{itemize}

Also, we provide some examples in man page. A demo verifies these
functions in different ways.
\begin{Command}
### Run the demo with 4 processors by
mpiexec -np 4 Rscript -e "demo(simple_pairs,'pbdMPI',ask=F,echo=F)"
\end{Command}
See \pkg{pbdDEMO} vignettes~\citep{Schmidt2013pbdDEMOvignette}
for more statistical examples.



\section[Long Vector and 64-bit for MPI]{Long Vector and 64-bit for MPI}
\label{sec:long_vector}
\addcontentsline{toc}{section}{\thesection. Long Vector and 64-bit for MPI}

\subsection{Long Vector for MPI}
The current \proglang{R} (3.1.0) uses \proglang{C} structure to extend
32-bit length limitation ($2^{31} - 1 = 2147483647$ defined as
\code{R_SHORT_LEN_MAX}) to 52-bit length ($2^{51} - 1 = 4503599627370496$
defined as \code{R_XLEN_T_MAX}).
In general, this is more portable and extensible when 128-bit integer
comming on (who know when the day comes ...)
However, a vector with elements
larger than $2^{31} - 1$ needs extra effort to be accessed in
\proglang{R}.
See ``R Internals'' for details.

The reason is that an integer is 4 bytes in both of
x86\_64 system (64-bit) and i386 system (32-bit).
Since the capacity of current machine and performance issues, there is no
benefit to use 8 bytes for integer.
In x86\_64 system, computers or compilers use either \code{long} or
\code{long long} for pointer address which is in \code{size_t} for unsigned
address or in \code{ptrdiff_t} for signed address.
For example, in GNU C (\code{gcc}), the flag \code{-m64} is to
use 4 bytes for \code{int} and 8 bytes for \code{long}
in x86\_64 system.

Therefore, the question is what are the differences of 64-bit and
32-bit system? One of them is ``pointer size''
which is 8 bytes in x86\_64 machine and it is 4 bytes in i386 machine.
This allows computer to lengthen memory and disk space.
Note that address is indexed by \code{long} or \code{long long} which is no
confilict with integer size, and 4 bytes integer is efficient and safe enough
for general purpose.
For example, \code{double *a} is a pointer (\code{a}) pointing to a real
scaler (\code{*a}), but the pointer's address (\code{\&a}) is in
\code{size_t} (\code{long} or \code{long long})
which is 8 bytes in x86\_64 system and is 4 bytes in i386 system.

% The other question is that is there a way to force an integer to be 8 bytes?
% The answer is yes for some compilers.
% For example, in GNU C (\code{gcc}), the flag \code{-m64} is to force compiler
% to compile code for x86\_64 system (4 bytes for \code{int} and 8 bytes for
% \code{long}), and the flag \code{-mint64} is to force
% compiler to use 8 bytes for integer.

To deal with long vector,
\pkg{pbdMPI} uses the same framework as \proglang{R} to build up MPI
collective functions. \pkg{pbdMPI} follows \proglang{R}'s standard to
assume a vector normally has length smaller than \code{R_SHORT_LEN_MAX}
which can be handled by most 32-bit functions. If the vector length
is greater than \code{R_SHORT_LEN_MAX}, then \proglang{R} names this as
long vector which also has the maximum \code{R_XLEN_T_MAX}. The vector length
is stored in type \code{R_xlen_t}. The \code{R_xlen_t} is \code{long} if
\code{LONG_VECTOR_SUPPORT} is defined, otherwise it is \code{int}.
\proglang{R} provides several \proglang{C} macro to check, access, and
manipulate the vector in \code{VECSXP} or general \code{SEXP}.
See \code{Rinternals.h} for details.

The \pkg{pbdMPI} first checks if the data size for
communication is greater than \code{SPMD_SHORT_LEN_MAX} or not.
If the data is long vector, then \pkg{pbdMPI} evokes collective functions
to send/receive chunk of data partitioned by \code{SPMD_SHORT_LEN_MAX} until
all chunks are all received/sent.
For some MPI collective functions such as
\code{allgather()} and \code{gather()}, extra space may be allocated
for receving chunks, then the chunks are copied to right memory address
by the rank of communicator from the extra space to the receiving buffer.

The reason is that most MPI collective functions rely on arguments
for indexing buff types and counting buffer sizes where the types and sizes
are both in \code{int}.
\code{SPMD_SHORT_LEN_MAX} is defined in \code{pbdMPI/src/spmd.h} and
usually is equal to \code{R_SHORT_LEN_MAX}. Developers may want to
use shorter length (such as \code{SPMD_INT8_LEN_MAX} which is $2^7 - 1 = 127$)
for testing without a large memory machine or
for debugging without recompiling \proglang{R} with shorter
\code{R_SHORT_LEN_MAX}.

In \pkg{pbdMPI},
the implemented MPI collective functions for long vector are
\code{bcast()}, \code{allreduce()},
\code{reduce()}, \code{send()}, \code{recv()}, \code{isend()},
\code{irecv()}, \code{allgather()}, \code{gather()}, and \code{scatter()}.
The other MPI collective functions are
``NOT'' implemented due to the complexity of memory allocation for long vector
including \code{allgatherv()}, \code{gatherv()}, \code{scatterv()},
\code{sendrecv()}, and \code{sendrecv.replace()}.


\subsection{64-bit for MPI}

The remaining question is that does MPI library support 64-bit system?
The answer is yes, but users may need to recompile MPI libraries for
64-bit support.
The same way as \proglang{R} to enable 64-bit system that MPI libraries
may have 8 bytes pointer in order to communicate larger memory or disk space.

For example, the OpenMPI provides next to check if 64-bit system is used.
\begin{Command}
ompi_info -a | grep 'int.* size: '
\end{Command}
If the output is
\begin{Command}
              C int size: 4
          C pointer size: 8
       Fort integer size: 8
      Fort integer1 size: 1
      Fort integer2 size: 2
      Fort integer4 size: 4
      Fort integer8 size: 8
     Fort integer16 size: -1
\end{Command}
then the OpenMPI supports 64-bit system.
Otherwise, users may use the next to reinstall OpenMPI as
\begin{Command}
./configure --prefix=/path_to_openmpi \
            CFLAGS=-fPIC \
            FFLAGS="-m64 -fdefault-integer-8" \
            FCFLAGS="-m64 -fdefault-integer-8" \
            CFLAGS=-m64 \
            CXXFLAGS=-m64
\end{Command}
and remember to reinstall \pkg{pbdMPI} as well.

Note that 64-bit pointer may only provide larger size of data, but
may degrade hugely for other computing. In general, communication with a
large amount of data is a very bad idea. Try to redesign algorithms to
communicate lightly such as via sufficient statistics, or to rearrange and
load large data partially or equally likely to every processors.
